 


%\subsubsection*{Objectives}
%\begin{itemize}
%	\item Calculate length of a vector
%	\item Calculate dot products
%	\item Use dot product to determine  
%\end{itemize}



\subsubsection*{Learning outcomes}
Be able to:
\begin{itemize}
	 % \item Identify a $m \times n$ matrix
	\item Use SVD to perform PCA
\end{itemize}





\rule[0.01in]{\textwidth}{0.0025in}
% ---------------------------------------------------- % 

\begin{example}
Given the following scores for a class of math students, we wish to determine which has the most impact on their overall score.

\begin{table}[h]
\begin{center}
\begin{tabular}{cccc}
\toprule
\textbf{Student} & \textbf{Homework} & \textbf{Midterm} & \textbf{Final Exam} \\
\midrule
S1 & 198 & 200 & 196  \\
S2 & 160 & 165 & 165  \\
S3 & 158 & 158 & 133  \\
S4 & 150 & 165 & 91  \\
S5 & 175 & 182 & 151  \\
S6 & 134 & 135 & 101  \\
S7 & 152 & 136 & 80  \\
\bottomrule
\end{tabular}
\caption{Scores for seven students}
\end{center}

\label{default}
\end{table}%



Proceed as follows:

\begin{enumerate}
	\item Find the mean of each column (i.e., mean of the midterm exam).  
	
	\item Subtract the mean from all entries in the column.  This is ``normalizing" the data.  This new matrix will be called $A$.
	
	\item Compute the covariance matrix:  $S = \frac{1}{n-1} A^TA$.  Interpret $S$.  What are the diagonal entries?  Off-diagonal entries?
	
	\item Find the eigenvalues and eigenvectors of $S$.  These will be used to find the singular values and right singular vectors (i.e., \textit{principle components}).  
	
	
\end{enumerate}	
\end{example}

%
%%
%%%						  
%%%% SECTION:  Matrices
%%%
%%
%
 

