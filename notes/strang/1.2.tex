%\section{Outline}
%\input{includes/thesis}
%Given a $m \times n$ system of equations: 
%\section*{Define arithmetic operations ($+, -, \times$) on matrices}
% \section*{Introduction}


%\subsubsection*{Objectives}
%\begin{itemize}
%	\item Calculate length of a vector
%	\item Calculate dot products
%	\item Use dot product to determine  
%\end{itemize}



\subsubsection*{Learning outcomes}
Be able to:
\begin{itemize}
	\item Calculate length of a vector. 
	\item Calculate dot products
	\item Normalize a vector
	\item Use dot product to determine if two vectors are perpendicular
	\item Find angle between two vectors
	\item Know commands for length and dot products in MATLAB
\end{itemize}





\rule[0.01in]{\textwidth}{0.0025in}
% ---------------------------------------------------- % 


%
%%
%%%						  
%%%% SECTION:  Dot Product
%%%
%%
%
\section{Dot Product}

% DEFINITION: Dot Product
\begin{tcolorbox}[colback=yellow!10!,colframe=gray!15!]
\begin{definition}[Dot Product or Inner product]
The \textbf{dot product} or \textbf{inner product} of ${\bf x} = (x_1, x_2)$ and  ${\bf y} = (y_1, y_2)$ is the number, 
\[ {\bf x} \cdot {\bf y} = x_1 \, y_1 + x_2 \, y_2 \]

Alternative notation for inner product of two vectors ${\bf x}$ and  ${\bf y}$ is 
\[ {\bf x} \cdot {\bf y} = \langle {\bf x}, {\bf y}  \rangle \]
The dot product is commutative, 
\[  {\bf x} \cdot {\bf y}  = {\bf y} \cdot {\bf x} \]
\end{definition}	 
\end{tcolorbox} 



% EXAMPLE
\begin{example}
	The dot product of  ${\bf x} =(4,2)$ and  ${\bf y} =(-1,2)$ is
	\[ 4(-1)+2 \cdot 2 = -4 + 4 = 0  \]
\end{example}

\rule[0.01in]{\textwidth}{0.0025in}
% ---------------------------------------------------- % 








% EXAMPLE
\begin{example}
	The prices of three products are $(p_1, p_2, p_3)$ for each unit (this is the price vector) and the quantities bought or sold (quantity vector) is $(q_1, q_2, q_3)$--positive when sold and negative when bought.  The dot product represents the total income.
	\[ \text{Income} = (q_1, q_2, q_3) \cdot (p_1, p_2, p_3)  = q_1p_1 + q_2p_2+q_3 p_3 \]
	
	
	\end{example}

\rule[0.01in]{\textwidth}{0.0025in}
% ---------------------------------------------------- % 











% DEFINITION: orthogonal
\begin{definition}
 Two vectors ${\bf x}$ and ${\bf y}$  are \textbf{orthogonal}, denoted ${\bf x} \perp {\bf y}$,  if ${\bf x} \cdot {\bf y} = 0$.   
\end{definition}

% ---------------------------------------------------- % 
\rule[0.01in]{\textwidth}{0.0025in}
% ---------------------------------------------------- % 


%
%%
%%%						  
%%%% SECTION:  Length of vector 
%%%
%%
%
\section{Length of vector}
% DEFINITION: Dot Product
\begin{tcolorbox}[colback=yellow!10!,colframe=gray!15!]
\begin{definition} 
The \textbf{length} of a vector ${\bf x}$ is the square root of the inner product of ${\bf x}$ with itself.  That is, 

\[ || {\bf x}|| = \sqrt{{\bf x} \cdot {\bf x}} = \left(\sum_{i=1}^n x_i^2\right)^{1/2} \]
\end{definition}	 
\end{tcolorbox} 



\begin{definition} 
A \textbf{unit vector} is a vector with length one.  That is, $ || {\bf x}|| = 1$. 
\end{definition}	 

% ---------------------------------------------------- % 
\rule[0.01in]{\textwidth}{0.0025in}
% ---------------------------------------------------- % 


% DEFINITION: Unit vector
\begin{definition} 
${\bf u} = {\bf x} / || {\bf x} ||$ is a unit vector in the direction of ${\bf x}$. 
\end{definition}	 

% ---------------------------------------------------- % 
\rule[0.01in]{\textwidth}{0.0025in}
% ---------------------------------------------------- % 

\begin{example}
If ${\bf x} = (1,1)$, then 
\[ {\bf u} = \frac{(1,1)}{\sqrt{2}} = \left(\frac{1}{\sqrt{2}}, \frac{1}{\sqrt{2}}\right) \] 
\end{example}
% ---------------------------------------------------- % 
\rule[0.01in]{\textwidth}{0.0025in}
% ---------------------------------------------------- % 


\begin{example}
There are unit vectors:  ${\bf i} = (1,0)$, ${\bf j} = (0,1)$, and ${\bf u} = (\cos \theta, \sin \theta)$.  

\begin{align*}
|| i || &= \sqrt{1^2+0^2} = 1\\
|| j || &= \sqrt{0^2+1^2} = 1\\
|| u || &= \sqrt{\sin^2 \theta + \cos^2 \theta} = 1
\end{align*}
\end{example}
% ---------------------------------------------------- % 
\rule[0.01in]{\textwidth}{0.0025in}
% ---------------------------------------------------- % 










\begin{theorem}
	 If ${\bf x} \perp {\bf y}$, then ${\bf x} \cdot {\bf y} = 0$.
	
	\begin{proof}
		Suppose ${\bf x} \perp {\bf y}$.  Geometrically, ${\bf x} -  {\bf y}$ is the vector connecting the tip of ${\bf x}$ to the tip of ${\bf y}$.  In this case, it is the hypothenuse.  
		From Pythagorean Law, 
		
		\begin{align}
		 ||{\bf x}||^2+ ||{\bf y}||^2 &= || {\bf x}-{\bf y}||^2 \\
		 (x_1^2+x_2^2) + (y_1^2+y_2^2) &=  (x_1-y_1)^2+(x_2-y_2)^2\\
		 (x_1^2+x_2^2) + (y_1^2+y_2^2)	&= x_1^2 - 2x_1y_1 + y_1^2 + x_2^2-2x_2 y_2 + y_2^2\\
								0 &=  - 2x_1y_1 -2x_2 y_2\\
								0&=x_1y_1 +x_2 y_2\\
								0 &= {\bf x} \cdot {\bf y}
		 \end{align}
		
	\end{proof}
\end{theorem}


Right angles produce dot product equal to 0.  If ${\bf x} \cdot {\bf y} > 0$, then $\theta < 90^\circ$.  If ${\bf x} \cdot {\bf y} < 0$, then $\theta > 90^\circ$.  




% ---------------------------------------------------- % 
\rule[0.01in]{\textwidth}{0.0025in}
% ---------------------------------------------------- % 


\begin{tcolorbox}[colback=yellow!10!,colframe=gray!15!]
\begin{formula}[Cosine Similarity]
If ${\bf x} $ and ${\bf y} $ are nonzero vectors then
 \[ \cos \theta = \frac{{\bf x} \cdot {\bf y}}{||{\bf x} || \, ||{\bf y}||}  \]
 \end{formula}	 
\end{tcolorbox} 








\begin{tcolorbox}[colback=yellow!10!,colframe=gray!15!]
\begin{theorem}[Cauchy-Schwarz Inequality]
If ${\bf x} $ and ${\bf y} $ are vectors (in an inner product space) then
 \[ |  {\bf x} \cdot {\bf y} | \le ||{\bf x}|| \,  ||{\bf y}|| \]
 \end{theorem}	 
\end{tcolorbox} 










\begin{tcolorbox}[colback=yellow!10!,colframe=gray!15!]
\begin{theorem}[Triangle Inequality]
If ${\bf x} $ and ${\bf y} $ are vectors  then
 \[ ||  {\bf x}  +  {\bf y} || \le ||{\bf x}||  +   ||{\bf y}|| \]
 \end{theorem}	 
\end{tcolorbox} 






\subsection*{MATLAB Commands}
\begin{itemize}
	\item Length of ${\bf v}$:  \\ 
	\texttt{norm(v)}
	\item Dot product of ${\bf u} \cdot {\bf v}$: \\
	\textsf{u' * v} 
\end{itemize}




% ---------------------------------------------------- % 
\rule[0.01in]{\textwidth}{0.0025in}
% ---------------------------------------------------- % 




\section*{Next time...}
Section 1.3: Matrices





\subsubsection*{Homework}
\textsection1.1: \#1, 2, 13, 16, 19


