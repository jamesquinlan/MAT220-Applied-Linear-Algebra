%\section{Outline}
%\input{includes/thesis}
%Given a $m \times n$ system of equations: 
%\section*{Define arithmetic operations ($+, -, \times$) on matrices}
\section*{Introduction}

Partitioning a matrix into  submatrix \textit{blocks} has broad utility and can lead to incredible insights \cite{carlsonmaa59}.  In particular, partitioning can be used in linear algebra proofs.   
\section*{Objectives}
\begin{enumerate}
	\item Introduce and define Block matrices
	\item State basic rules for Block multiplication   
	 \end{enumerate}



 

\begin{example} Partition a matrix using horizontal and vertical lines.  The example below uses only one horizontal and one vertical.
\[ \begin{bmatrix}[cc|c]  
  	1  &   1 &    2 \\
     	3  &    2  &    1\\
	\hline
     	0  &    1   &  3
     \end{bmatrix}  \]	
\end{example}
 
\begin{example} Partitioned matrix into nine blocks:
\[ A = \begin{bmatrix}[cc|cc|c]  
  	1  &   1 &    2 & 4 & 5 \\
     	3  &    2  &    1 & 3 & 1 \\
	\hline
     	0  &    1   &  3 & 7 & 2 \\ 
	2  &    1   &  1 & 1 & 0 \\ 
	\hline
	1  &   3   &  0 & 1 & 1
     \end{bmatrix}  = \begin{bmatrix} A_{11} &  A_{12} &  A_{13} \\  A_{21} &  A_{22} &  A_{23} \\  A_{31} &  A_{32} &  A_{33} \end{bmatrix} \]	
\end{example}


  

\rule[0.01in]{\textwidth}{0.0025in}
% ---------------------------------------------------- % 


 

 
%\rule[0.01in]{\textwidth}{0.0025in}
% ---------------------------------------------------- % 

\section*{Block Multiplication}
Let $A$ be an $m \times n$ matrix and $B$ an $n \times p$. 

\begin{enumerate}
	\item If $B = \begin{bmatrix} B_1 &	B_2 \end{bmatrix}$ where $B_1$ is an $n \times k$ matrix and $B_2$ is an $n \times (p-k)$ matrix, then
	\begin{align*}
	 AB 	&= A ({\bf b}_1, \dots, {\bf b}_k, {\bf b}_{k+1}, \dots, {\bf b}_p) \\
	 	&= (A{\bf b}_1, \dots, A{\bf b}_k, A{\bf b}_{k+1}, \dots, A{\bf b}_p)\\
		&= (A({\bf b}_1, \dots, {\bf b}_k), A({\bf b}_{k+1}, \dots, {\bf b}_p))\\
		&= \begin{bmatrix} AB_1	& 	AB_2 \end{bmatrix}
	\end{align*}
	
	Thus, the matrix $A$ acts like a scalar,
	
	\begin{tcolorbox}[colback=yellow!10!,colframe=gray!15!]
	\[
	AB = A \begin{bmatrix} B_1	& 	B_2 \end{bmatrix} =  \begin{bmatrix} AB_1	& 	AB_2 \end{bmatrix} 
	\]
	\end{tcolorbox}
	
	
\rule[0.01in]{\textwidth}{0.0025in}
% ---------------------------------------------------- % 

	
	% case 2
	\item Let $A_1$ be a $k \times n$ matrix and $A_2$ is an $(m - k) \times n$ matrix.  If $A = \begin{bmatrix} A_1 \\ A_2 \end{bmatrix}$
	then 
	
 	\[ AB = \begin{bmatrix} A_1 \\ A_2 \end{bmatrix} B = \begin{bmatrix} \vec{{\bf a}}_1  \\ \vdots   \\ \vec{{\bf a}}_k \\ \hline \vec{{\bf a}}_{k+1} \\ \vdots \\  \vec{{\bf a}}_m  \end{bmatrix} B = \begin{bmatrix} \vec{{\bf a}}_1 B  \\ \vdots   \\ \vec{{\bf a}}_k B \\ \hline \vec{{\bf a}}_{k+1} B \\ \vdots \\  \vec{{\bf a}}_m B  \end{bmatrix} =  \begin{bmatrix}  \begin{bmatrix} \vec{{\bf a}}_1   \\ \vdots   \\ \vec{{\bf a}}_k  \end{bmatrix}  B  \\  \\  \begin{bmatrix} \vec{{\bf a}}_{k+1}  \\ \vdots \\  \vec{{\bf a}}_m  \end{bmatrix}  B \end{bmatrix}  = \begin{bmatrix} A_1 B \\ A_2 B \end{bmatrix} \]
 	
	
	
	
	
	\begin{tcolorbox}[colback=yellow!10!,colframe=gray!15!]
	\[ AB = \begin{bmatrix} A_1 \\ A_2 \end{bmatrix} B = \begin{bmatrix} A_1 B \\ A_2 B \end{bmatrix} \]
	\end{tcolorbox}
	
	
\rule[0.01in]{\textwidth}{0.0025in}
% ---------------------------------------------------- % 

	
	\item Let $A = \begin{bmatrix} A_1 & A_2 \end{bmatrix}$ where $A_1$ be  a $m \times r$ matrix,  $A_2$ is an $m \times (n - r)$ matrix, and $B = \begin{bmatrix} B_1 \\ B_2 \end{bmatrix}$ matrix with $B_1$ an $r \times q$ matrix and $B_2$ a $(n - r) \times q$ matrix.  Then,     
	
		\begin{tcolorbox}[colback=yellow!10!,colframe=gray!15!]
$$AB = \begin{bmatrix} A_1 & A_2 \end{bmatrix} \begin{bmatrix} B_1 \\ B_2 \end{bmatrix} = A_1B_1 +  A_2B_2$$
	\end{tcolorbox}



\rule[0.01in]{\textwidth}{0.0025in}
% ---------------------------------------------------- % 





\item In general, if the blocks have the proper dimensions, then block multiplication can be carried out in the same manner as ordinary  matrix multiplication.  In particular, if
\[ A =  \begin{bmatrix}  A_{11} & \dots & A_{1n}\\ \vdots & & \vdots \\ A_{m1} & \dots & A_{mn} \end{bmatrix}  \text{\;\;\; and \;\;\;} B =  \begin{bmatrix}  B_{11} & \dots & B_{1p}\\ \vdots & & \vdots \\ B_{n1} & \dots & B_{np} \end{bmatrix}  \]
then
		\begin{tcolorbox}[colback=yellow!10!,colframe=gray!15!]

\[ AB =  \begin{bmatrix}  C_{11} & \dots & C_{1p}\\ \vdots & & \vdots \\ C_{m1} & \dots & C_{mp} \end{bmatrix} \]
where
\[   C_{ij} = \sum_{k=1}^n  A_{ik}B_{kj} \]
	\end{tcolorbox}



\rule[0.01in]{\textwidth}{0.0025in}
% ---------------------------------------------------- % 


\item $A B = A \begin{bmatrix} {\bf b}_1 & \dots & {\bf b}_k \end{bmatrix} = (A{\bf b}_1, A{\bf b}_2, \dots, A {\bf b}_k)$.




\rule[0.01in]{\textwidth}{0.0025in}
% ---------------------------------------------------- % 


\end{enumerate}
 
 
 
 \begin{theorem}
 If $A$ is an $n \times n$ block diagonal matrix of the form
 \[  A = \begin{bmatrix}  A_{11}   &  O \\ O  &  A_{22}  \end{bmatrix} \]
 where $A_{11}$ is  $k \times k$ ($k<n$) and $A_{22}$ is $(n-k) \times (n-k)$, then $A$ is nonsingular if and only if $A_{11}$ and $A_{22}$ are nonsingular.\
 
 \begin{proof}
 	%  -->
 	($\Rightarrow$)   Assume $A$ is nonsingular and let $B = A^{-1}$.  Partition $B$ in same way as $A$, then . 
	 \[   \begin{bmatrix}  A_{11}^{-1}   &  O \\ O  &  A_{22}^{-1}  \end{bmatrix}  \begin{bmatrix}  B_{11}   &  B_{12} \\ B_{21}  & B_{22}  \end{bmatrix}  =  \begin{bmatrix}  I_{k}   &  O \\ O  &  I_{n-k}  \end{bmatrix}   \]

	and
	 
	 \[   \begin{bmatrix}  B_{11}   &  B_{12} \\ B_{21}  & B_{22}  \end{bmatrix}   \begin{bmatrix}  A_{11}^{-1}   &  O \\ O  &  A_{22}^{-1}  \end{bmatrix}  =  \begin{bmatrix}  I_{k}   &  O \\ O  &  I_{n-k}  \end{bmatrix}   \]
	
	Using \#4 above we have
	\[  B_{11}  A_{11}^{-1}  = I_k  =  A_{11}^{-1}B_{11}    \text{\;\;\;\;  and \;\;\;\; } B_{22} A_{22}^{-1}  =  I_{n-k}  = A_{22}^{-1} B_{22} \]
	
	
	
	Hence, $A_{11}$ and $A_{22}$  are both nonsingular with inverses $B_{11}$ and $B_{22}$, respectively.
	
		% <--
	($\Leftarrow$) Suppose $A_{11}$ and $A_{22}$ are nonsingular, then 
	\[   \begin{bmatrix}  A_{11}   &  O \\ O  &  A_{22}  \end{bmatrix}  \begin{bmatrix}  A_{11}^{-1}   &  O \\ O  &  A_{22}^{-1}  \end{bmatrix}  =  \begin{bmatrix}  I_{k}   &  O \\ O  &  I_{n-k}  \end{bmatrix} = I  \]
	
	and 
	\[  \begin{bmatrix}  A_{11}^{-1}   &  O \\ O  &  A_{22}^{-1}  \end{bmatrix}   \begin{bmatrix}  A_{11}   &  O \\ O  &  A_{22}  \end{bmatrix}    =  \begin{bmatrix}  I_{k}   &  O \\ O  &  I_{n-k}  \end{bmatrix} = I  \]
Therefore, 
\[  A^{-1} = \begin{bmatrix}  A_{11}^{-1}   &  O \\ O  &  A_{22}^{-1}  \end{bmatrix}  \]	 
 \end{proof}
 \end{theorem}



\rule[0.01in]{\textwidth}{0.0025in}
% ---------------------------------------------------- % 

























%\section*{Summary}


 %In this section we 
%\begin{enumerate}
%	\item Introduced and defined elementary matrices
%	\item Enumerated three equivalent conditions for nonsingularity 
%	\item Defined and discussed triangular (upper \& lower) and diagonal matrices
%	\item Used the inverse of the product of a finite sequence of elementary matrices in part of the factorization of %matrix $A$ 
	
	
% \end{enumerate}
 



\section*{Next time...}
Section 2.1 - 2.3:  Determinants

