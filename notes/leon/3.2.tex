%\section{Outline}
%\input{includes/thesis}
%Given a $m \times n$ system of equations: 
%\section*{Define arithmetic operations ($+, -, \times$) on matrices}
\section*{Introduction}

Often a subset of vectors from a vector space form a vector spaces itself.  If so, it is called a \textbf{subspace}.


 
 
  

\rule[0.01in]{\textwidth}{0.0025in}
% ---------------------------------------------------- % 


 

 
%\rule[0.01in]{\textwidth}{0.0025in}
% ---------------------------------------------------- % 

\section*{Subspace}
\begin{definition}


If $S$ is a nonempty subset of a vector space $V$, and $S$ satisfies the conditions: \\



	\begin{tcolorbox}[colback=yellow!10!,colframe=gray!15!]

\begin{enumerate}
	\item[(i)] $\alpha {\bf} \in S$ whenever ${\bf x} \in S$ for any scalar $\alpha$
	\item[(ii)] ${\bf x} + {\bf y} \in S$ whenever ${\bf x} \in S$ and ${\bf y} \in S$.
\end{enumerate}
	 
	\end{tcolorbox}
	
\end{definition}
	
\rule[0.01in]{\textwidth}{0.0025in}

\textbf{To show a subset $S$ is a subspace}:
\begin{enumerate}
	\item Show it is nonempty.  The easiest way is to show that ${\bf 0} \in S$.
	\item Show it is closed under addition (i.e., ${\bf x} + {\bf y} \in S$)
	\item Show it is closed under scalar multiplication, (i.e., $\alpha {\bf x} \in S$ for all $\alpha \in F$ and for all ${\bf x} \in S$.)
\end{enumerate}


\rule[0.01in]{\textwidth}{0.0025in}
% ---------------------------------------------------- % 













% ---------------------------------------------------- % 
\begin{example}
$S = \{ (x_1, x_2, x_3)  \in  \mathbb{R}^3 \; : \; x_1=x_2 \}$.
 \end{example}
	
	 
\rule[0.01in]{\textwidth}{0.0025in}
% ---------------------------------------------------- % 






% ---------------------------------------------------- % 
\begin{example}
$S = \{ A \in M_2(\mathbb{R}) \; : \; a_{12} = -a_{21} \}$.
 \end{example}
	
	 
\rule[0.01in]{\textwidth}{0.0025in}
% ---------------------------------------------------- % 


 
 
 
 
 
 

% ---------------------------------------------------- % 
\begin{example}
$S = \{ p(x) \in P_n \; : \; p(0) = 0 \}$.
 \end{example}
	
	 
\rule[0.01in]{\textwidth}{0.0025in}
% ---------------------------------------------------- % 


 
 
 
 
 
 
 

% ---------------------------------------------------- % 
\begin{example}
$C^n[a,b] \subset C[a,b]$ (the set of all  functions that have continuous $n$th derivative.
 \end{example}
	
	 
\rule[0.01in]{\textwidth}{0.0025in}
% ---------------------------------------------------- % 








% ---------------------------------------------------- % 
\begin{example}
Let $S$  be the set of all $f \in C^2[a,b]$ such that 
\[  f''(x) +f(x) = 0 \]

 \end{example}
	
	 
\rule[0.01in]{\textwidth}{0.0025in}
% ---------------------------------------------------- % 


 


  
  
  
  
  
  
  \section*{Null Space}
  
  
  \begin{definition}
Let $A$ be an $m \times n$ matrix.  The set of solutions to the homogenous system is a subspace called the \textbf{null space}.



	\begin{tcolorbox}[colback=yellow!10!,colframe=gray!15!]

\[  N(A) = \{ {\bf x} \in \mathbb{R}^n \; | \;  A{\bf x} = {\bf 0}  \} \]
	 
	\end{tcolorbox}
	
\end{definition}
	
\rule[0.01in]{\textwidth}{0.0025in}

\textbf{To find the null space}:

Solve the system $A{\bf x} = {\bf 0} $ for ${\bf x}$.

\rule[0.01in]{\textwidth}{0.0025in}
% ---------------------------------------------------- % 






% ---------------------------------------------------- % 
\begin{example}
Find the null space of 
\[  
A = \begin{bmatrix}
1 & 1 & 1 & 0\\ 2 & 1 & 0 & 1
\end{bmatrix}
 \]

 \end{example}
	
	 
\rule[0.01in]{\textwidth}{0.0025in}
% ---------------------------------------------------- % 


 








%%%%%%%%%%%%%%%%% 

\section{The Span}

  \begin{definition}
Let ${\bf v}_1, {\bf v}_2, \dots, {\bf v}_n  \in V$.  The sum of the form $\alpha_1 {\bf v}_1 +  \alpha_2 {\bf v}_2 +  \cdots + \alpha_n {\bf v}_n$ is called a \textbf{linear combination}.  The set of all linear combinations of ${\bf v}_1, {\bf v}_2, \dots, {\bf v}_n$ is called the \textbf{span} of ${\bf v}_1, {\bf v}_2, \dots, {\bf v}_n$.



	\begin{tcolorbox}[colback=yellow!10!,colframe=gray!15!]

\[ \text{Span}({\bf v}_1, {\bf v}_2, \dots, {\bf v}_n) = \left\{ \sum_{i=1}^n (\alpha)_i {\bf v}_i \; | \; \forall  \alpha \in F    \right\}  \]
	 
	\end{tcolorbox}
	
\end{definition}
	
\rule[0.01in]{\textwidth}{0.0025in}







\begin{theorem}
	if ${\bf v}_1, {\bf v}_2, \dots, {\bf v}_n  \in V$, then $ \text{Span}({\bf v}_1, {\bf v}_2, \dots, {\bf v}_n)$ is a subspace of $V$.
	
	\begin{proof}
		\begin{enumerate}
			\item[(i)] Clearly it is nonempty
			 
			
			%${\bf 0} \in \text{Span}({\bf v}_1, {\bf v}_2, \dots, {\bf v}_n)$ since 
			% \[ 0 {\bf v}_1 +  0 {\bf v}_2 +  \cdots + 0 {\bf v}_n = {\bf 0}  \]
			
			\item[(ii)]  Let $\beta$ be a scalar and ${\bf v} = \alpha_1 {\bf v}_1 +  \alpha_2 {\bf v}_2 +  \cdots + \alpha_n {\bf v}_n$ in the span.    Then, 
			\[ \beta {\bf v} = (\beta \alpha_1) {\bf v}_1 + (\beta  \alpha_2) {\bf v}_2 +  \cdots + (\beta  \alpha_n) {\bf v}_n	\]
	which is also a linear combination of the ${\bf v}_i$'s and so in the Span.
	
	\item[(iii)]  Let ${\bf v}$ and ${\bf w}$ be elements in the Span.  Therefore,  $ {\bf v} = \alpha_1 {\bf v}_1 +  \alpha_2 {\bf v}_2 +  \cdots + \alpha_n {\bf v}_n$ and $ {\bf w} = \beta_1 {\bf v}_1 +  \beta_2 {\bf v}_2 +  \cdots + \beta_n {\bf v}_n$.  Consider, ${\bf v}+{\bf w}$, 
	\begin{align*}
		{\bf v}+{\bf w} &=  (\alpha_1 {\bf v}_1 +  \alpha_2 {\bf v}_2 +  \cdots + \alpha_n {\bf v}_n) +  ( \beta_1 {\bf v}_1 +  \beta_2 {\bf v}_2 +  \cdots + \beta_n {\bf v}_n\\
		&=  (\alpha_1+\beta_1) {\bf v}_1  +  \cdots + (\alpha_n + \beta_n) {\bf v}_n\\
		& \in  \text{Span}({\bf v}_1, {\bf v}_2, \dots, {\bf v}_n)
	\end{align*}
	
	\end{enumerate}
	\end{proof}
\end{theorem}
 
 \rule[0.01in]{\textwidth}{0.0025in}

 






%%%%%%
\section*{Spanning set for Vector Space}
If $\text{Span}({\bf v}_1, {\bf v}_2, \dots, {\bf v}_n) = V$, then ${\bf v}_1, \dots, {\bf v}_n$ is a \textbf{spanning set} for $V$.

\textbf{To determine if a set is a spanning set}
\begin{enumerate}
	\item Set up system of equation with the possible spanning vectors as columns of a matrix and an arbitrary vector $ {\bf b}$ from the space as the right hand side.  
	\[ A {\bf x} = {\bf b} \]
	\item Solve the system.  If consistent, then the set is a spanning set.
\end{enumerate}

 \rule[0.01in]{\textwidth}{0.0025in}




 \begin{example}
 Which of the following are spanning sets for $\mathbb{R}^3$?
 \begin{enumerate}
 	\item $\{ (1, 1, 1)^T, (1, 1, 0)^T, (1, 0, 0)^T \}$
	\item $\{ (1, 0, 1)^T, (0, 1, 0)^T \}$
	\item $\{(1, 2, 3)^T, (2, 1, 3)^T, (4, -1, 1)^T \}$
 \end{enumerate}
 \end{example}

 \rule[0.01in]{\textwidth}{0.0025in}


\begin{example}
	Do the vectors $1 - x^2, x+2$, and $x^2$ span $P_3$?
\end{example}

 \rule[0.01in]{\textwidth}{0.0025in}





\begin{theorem}
	If ${\bf x}_0$ is a solution of  $A {\bf x} = {\bf b}$, then ${\bf y}$ is a solution if and only if ${\bf y} = {\bf x}_0 + {\bf z}$ where ${\bf z} \in N(A)$.
	
	\begin{proof}
	($ \Leftarrow$) Suppose ${\bf y} = {\bf x}_0 + {\bf z}$ where ${\bf z} \in N(A)$.  Consider, 
	$ A {\bf y} = A {\bf x}_0 + A {\bf z} = A {\bf x}_0 + {\bf 0} = {\bf b}$.  Therefore, ${\bf y}$ is a solution.
	
	($ \Rightarrow$) Next, suppose $ A {\bf y} =  {\bf b}$, then we have that $A {\bf y}  - A {\bf x}_0 = {\bf b} -  {\bf b}  = {\bf 0}$.  Therefore, ${\bf y}  - {\bf x}_0 \in N(A)$.  Define ${\bf z } = {\bf y} - {\bf x}_0$, and solve for ${\bf y}$.        
	\end{proof}
\end{theorem}




\section*{Next time...}
Section 3.3: Linear Independence

